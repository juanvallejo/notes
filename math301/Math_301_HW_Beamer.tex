\documentclass{beamer}
\usepackage{verbatim, graphicx}
\usetheme{Copenhagen}


\begin{document}
\begin{frame}
  \title{Math 301: Function Theory}
  \author{Juan Vallejo}
  \maketitle
\end{frame}

% slide 0
\begin{frame}[fragile]\frametitle{What Are Functions?}
\begin{center}\textbf{Definition} \end{center}

\begin{itemize}
  \item In mathematics, a function is a relation between a set of inputs and a set of permissible outputs with the property that each input is related to exactly one output. %No comments to display here
  
  \vfill
  \begin{center}\includegraphics[scale=0.5]{Function_IO_sl2.png} \end{center}
  
  \vfill
  \item General format of functions:
  
  \vfill
  \begin{equation*}
    f(x) = x^2
  \end{equation*}
  \vfill

\end{itemize}

\end{frame}

% slide 1
\begin{frame}[fragile]\frametitle{History of Functions}
\begin{center} Concept emerged in the 17th century in connection with the development of calculus. For example the slope $\frac{dy}{dx}$ of a graph at a point was regarded as a function of the x-coordinate of the point. \end{center}

\begin{itemize}
  \item Went from being defined as \textit{analytic expression} to \textit{single-valued mapping from one set to another}.
  \item Term \textit{function} was introduced by \textbf{Gottfried Leibniz} in 1673 to describe a quantity related to a curve, such as a curve's slope at a specific point.
  \item Later re-defined by Leibniz as \textit{any expression made up of a variable and some constants}.
  \item The familiar notation $f(x)$ was later introduced by \textit{Leonhard Euler}.
  \item Functions from those times are called today \textit{differentiable functions}.
  
\end{itemize}

\end{frame}

% slide 1.5
\begin{frame}[fragile]\frametitle{Function Composition}
\begin{center} Composition is the operation of taking the output from one function and using that as the input to a second function. It, together with the algebraic operations on functions, is what allows us to take the basic building blocks of ``pure" function types and build large varieties of functions which is the focus of calculus. \end{center}

\begin{center} \textbf{Definition} \end{center}
\begin{itemize}
  \item The composition $f \circ g$ of the two functions $f$ and $g$ is the function which feeds an input to $g$ and feeds the output of $g$ to $f$
\end{itemize}

\begin{center}\includegraphics[scale=0.35]{Function_FuncComp_slOther.png} \end{center}

\end{frame}

% slide 2
\begin{frame}[fragile]\frametitle{Why Functions?}
%\begin{center}\textbf{Definition} \end{center}

\begin{itemize}
  \item They are an integral part of mathematics and computational world
    \begin{itemize}
      \item Theory is foundation for functional programming paradigm seen in languages such as Haskell and Python
    \end{itemize}
  \item Allow for use of rules more than once in an expression
  \item Allow for different parameter values
  \item Useful for recognizing similarities in a set of observations

\end{itemize}

\end{frame}

% slide 3
\begin{frame}[fragile]\frametitle{Components $-$ Domain and Range}
%\begin{center}\textbf{Definition} \end{center}

Functions can be seen as relations that uniquely associate members of one set with members of another set. \textit{A function from $A \to B$} is an object $f$ such that every $\alpha \in A$ is uniquely associated with an object $f(\alpha) \in B$. A function is therefore a many-to-one relation.

\vfill
\begin{itemize}
  \item The set $A$ of values at which a function is defined is called its domain
  \item The set $f(A) \subset B$ of values that the function can produce is called the \textit{range}. It is denoted as follows:
  \begin{equation*}
    D(f) = \lbrace x \in \mathbb{R}; f(x) \rbrace
  \end{equation*}
  \item In other words, range of a function is the set of all values obtained by substituting arguments from its domain

\end{itemize}

\end{frame}

% slide 4
\begin{frame}[fragile]\frametitle{Components $-$ Real Functions}
%\begin{center}\textbf{Definition} \end{center}

A function is defined as ``Real" when $f: A \subset \mathbb{R} \in \mathbb{R}$
\vfill
\begin{itemize}
  \item Domain elements in real functions are mapped to unique range elements
  \begin{itemize}
    \item Can be expressed graphically by performing the \textit{vertical line test}
  \end{itemize}
  \item They are the most important type of \textit{mapping}.
  \begin{itemize}
    \item Can be thought of as any mapping from some subset of the set of real numbers to the set of real numbers
  \end{itemize}
  
\end{itemize}

\end{frame}

% slide 5
\begin{frame}[fragile]\frametitle{Components $-$ Real Functions}
\begin{center}\textbf{Basic Notions of Real Functions} \end{center}

\vfill
\begin{itemize}
  \item Can be thought of as a perscription which assigns values to arguments
  \begin{itemize}
    \item For example: $y = f(x)$ means that to the value $x$ of the argument, the function $f$ assigns the value $y$
    \item Second notation: $f:x \to y$ implies that the function $f$ sends the value $x$ to $y$
  \end{itemize}
  \item Usual way of specifying assignments is that the function value $y$ can be obtained by substituting $x$ to a specific formula:
  \begin{equation*}
    f(x) = 2x + 3
  \end{equation*}
  
\end{itemize}

\end{frame}

% slide 6
\begin{frame}[fragile]\frametitle{Components $-$ Real Functions}
\begin{center}\textbf{Substitution and Visualization} \end{center}

\vfill
\begin{center} $f(a) = f \vert_a  =  f \vert_{x=a}$ \end{center}

\vfill
\begin{itemize}
  \item Used when function needs adjustment, but not yet ready for substitution. Example:
  \begin{equation*}
    \left.\frac{x^2 - 1}{x-1}\right\rvert_{x=3} = \left.\frac{(x-1)(x+1)}{x-1}\right\rvert_{x=3} = 4.
  \end{equation*}
  \item Graphs are common way of visualizing functions in mathematics.
  \item Used to mark in the two-dimensional plane $(x,y)$ all couples $x, f(x)$
  
\end{itemize}

\end{frame}

% slide 7
\begin{frame}[fragile]\frametitle{Types of Functions}

\vfill
\begin{center}\textbf{Functions are primary objects of study in calculus, consisting of various types:} \end{center}

\vfill
\begin{itemize}
\item linear
\item polynomial
\item trigonometric
\item logarithmic
\item exponential
\item rational
\item piece-wise
\item power
\end{itemize}

\end{frame}

% slide 8
\begin{frame}[fragile]\frametitle{Linear Functions}
% \begin{center}\textbf{Substitution and Visualization} \end{center}

\vfill
Allow us to approximate more complicated functions in differential calculus. There are three standad forms of linear functions:

\vfill
\begin{align*}
  f(x) &= mx + b &\,\text{(Slope-intercept form)} \\
  y - y_o &= m(x - x_0) &\, \text{(The ``point-slope" form)} \\
  Ax + By &= C &\, \text{(The ``general" form)}
\end{align*}

\end{frame}

% slide 9
\begin{frame}[fragile]\frametitle{Linear Functions}
\begin{center}\textbf{Frequently used to solve for the following calculations:} \end{center}

\vfill
\begin{itemize}
  \item \textbf{Intercepts}
  \begin{itemize}
    \item Finding the slope, x-intercept, and y-intercept of the line given its equation
  \end{itemize}
  \item \textbf{Intersections}
  \begin{itemize}
    \item Finding the intersection of two lines from their equations
  \end{itemize}
  \item \textbf{Equations}
  \begin{itemize}
       \item Finding the equation of the line through a series of points given one of the following three pairs of data:
       \begin{itemize}
        \item the slope of the line and the y-intercept
        \item the slope of the line and a point $(x_0,y_0)$ on the line
        \item the coordinates of two points on the line
       \end{itemize}
  \end{itemize}

\end{itemize}

\end{frame}

% slide 10
\begin{frame}[fragile]\frametitle{Linear Functions}
\begin{center}\textbf{Graphs} \end{center}

If $f(x)$ is linear, the graph of $y = f(x)$ is a straight line. The parameter $m$ in the formula below represents the slope of a line. In general form, the slope is $-A/B$ if $B \neq 0$, and indefinite if $B=0$.
\vspace{0.1in}

\begin{center}\includegraphics[scale=0.08]{Function_Graph_sl10.png} \end{center}

\end{frame}

% slide 11
\begin{frame}[fragile]\frametitle{Polynomial Functions}
\begin{center} Can be represented by the formula: \end{center}
\begin{align*}
  f(x) = a_nx^n + a_{n-1}x{n-1} + \cdots a_1x + a_0
\end{align*}
where $a_0, a_1$, \dots , $a_n$ are real numbers. These are called the coefficients. It is assumed that $a_n \neq 0$. The number n is called the degree of the polynomial function.
\vspace{0.1in}

\end{frame}

% slide 12
\begin{frame}[fragile]\frametitle{Polynomial Functions}
\begin{center} \textbf{Standard Forms} \end{center}

\begin{equation*}
  f(x) = a_nx^n + a_{n-1}x^{n-1}+ \cdots + a_1x+a_0
\end{equation*}

\vfill
\begin{itemize}
  \item The $a_i$ are real numbers and are called  \textbf{coefficients}.
  \item The term $a_n$ is assumed to be non-zero and is called the \textbf{leading term.}
  \item The degree of the polynomial is the largest exponent of $x$
  \item A Polynomial with one term is called a \textbf{monomial}
  \item A degree-zero polynomial is a constant
  \item A degree $1$ polynomial is a linear function
  \item A degree $2$ polynomial is a quadratic function
  \item a degree $3$ polynomial is a cubic
\end{itemize}

\end{frame}

% slide 13
\begin{frame}[fragile]\frametitle{Power Functions}
\begin{center} Power functions are of the form: \end{center}

\begin{equation*}
  f(x) = kx^p
\end{equation*}
\begin{center} where $p$ is any real number and $k$ is non-zero. Rules are: \end{center}

\vfill
\begin{itemize}
  \item $x^{(p+q)} = (x^p)(x^q)$
  \item $x^{pq} = (x^p)^q$
  \item $x^{-p} = \frac{1}{x^p}$
  \item $x^{\frac{1}{p}}$ is the $p^{th}$ root of x
  \item $x^0 = 1$ for any $x \neq 0$. $0^0$ is undefined
  \item $(xy)^p = (x^p)(y^p)$
\end{itemize}

\end{frame}

% slide 14
\begin{frame}[fragile]\frametitle{Power Functions}
\begin{center}Graphs of Power Functions\end{center}

\begin{center}\includegraphics[scale=0.6]{Function_PowerGraph_sl14.png} \end{center}

\end{frame}

% slide 15
\begin{frame}[fragile]\frametitle{Rational Functions}
\begin{center} Functions that can be represented as the quotient of polynomials \end{center}

\begin{itemize}
  \item Typical form of $\frac{p(x)}{a(x)}$ where $p$ and $q$ are polynomials.
  \item $p(x)$ is called the numerator and $q(x)$ is the denominator
\end{itemize}

\vfill
\begin{equation*}
  g(x) = \frac{x^2 - 4}{x^2 - 5x + 6}
\end{equation*}
In the example above, $x^2 - 4$ is the numerator and $x^2 - 5x + 6$ is the denominator. A polynomial is a rational function whose denominator is $1$.
% \begin{center}\includegraphics[scale=0.6]{Function_PowerGraph_sl14.png} \end{center}

\end{frame}

% slide 16
\begin{frame}[fragile]\frametitle{Rational Functions}
\begin{center} Domain and Range of Rational Functions \end{center}

\begin{itemize}
  \item Domain of a rational function $\frac{p(X)}{q(x)}$ consists of all ponints where $q(x)$ is non-zero
  \item Domain depends on the way in which $p(x)$ and $q(x)$ are chosen. By expanding the function above, $x = 2$ will now appear in its domain, whereas it did not before:
\end{itemize}

\begin{equation*}
  g(x) = \frac{(x-2)(x + 2)}{(x - 3)(x - 2)}
\end{equation*}
% \begin{center}\includegraphics[scale=0.6]{Function_PowerGraph_sl14.png} \end{center}

\end{frame}

% slide 17
\begin{frame}[fragile]\frametitle{Other Functions}

\begin{itemize}
  \item Exponential Functions
  \begin{itemize}
    \item Functions of the form $f(x) =  b^x$ for any positive real number $b$.
    \item Characterized by their rate of growth being proportional to their value
  \end{itemize}
\end{itemize}
\begin{center}\includegraphics[scale=0.4]{Function_ExpoGraph_slOther.png} \end{center}

\end{frame}

% slide 18
\begin{frame}[fragile]\frametitle{Other Functions}

\begin{itemize}
  \item Logarithmic Functions
  \begin{itemize}
    \item Defined as any function of the form $\log_bx$ being the inverse function of the exponential function $b^x$
    \item Three bases for logarithms:
    \begin{itemize}
      \item \textbf{Natural logarithms} written as $\ln x$, expressed as the logarithm to the base $e$. Thus $e^x$ and $\ln(x)$ are inverse functions
      \item \textbf{Common logarithms} written as $\log x$, expressed as the logarithm to the base $10$. Thus $10^{\log x} = x$
      \item \textbf{Binary logarithms} used in communications and computer science with a base of $2$. Expressed as $\log(x)$ or $\log_2x$ where $x$ is an integer measuring the number of bits it takes to write $x$.
    \end{itemize}
  \end{itemize}
\end{itemize}
\begin{center}\includegraphics[scale=0.25]{Function_LogGraph_slOther.png} \end{center}

\end{frame}

% slide 19
\begin{frame}[fragile]\frametitle{Other Functions}

\begin{itemize}
  \item Trigonometric Functions
  \begin{itemize}
    \item Examples of these include $\sin(x)$ and $\cos(x)$ defined in the image below:
      \begin{center}\includegraphics[scale=0.45]{Function_TrigCircl_slOther.png} \end{center}
    \item Other trigonometric functions defined below:
    \begin{itemize}
      \item $\tan(x) = \frac{\sin(x)}{\cos(x)}$
      \item $\cot(x) = \frac{\cos(x)}{\sin(x)}$
      \item $\sec(x) = \frac{1}{\cos(x)}$
      \item $\csc(x) = \frac{1}{\sin(x)}$
    \end{itemize}
  \end{itemize}
\end{itemize}
\begin{center}\includegraphics[scale=0.3]{Function_SinCos_slOther.png} \end{center}

\end{frame}

% slide 20
\begin{frame}[fragile]\frametitle{Real-World Applications}
\begin{itemize}
  \item Mathematical function theory is used in the design of various programming languages. The intent of using such concept as a basis in a different approach to the design of programming languages gives way to several advantages and disadvantages not seen with procedural and object-oriented programming. Examples of purely functional programming languages include:
  \begin{itemize}
    \item{Haskell}
    \item{Agda}
    \item{Mercury}
  \end{itemize}
  \vfill
  \item It is further used in every-day life when computing things such as \textbf{miles per gallon}, or \textbf{weekly salaries}, or \textbf{compound interest}. These are all functions of specific rates such as houerly pay for salaries, or initial investment and interest rate over time for compound interest.
\end{itemize}
% \begin{center}\includegraphics[scale=0.6]{Function_PowerGraph_sl14.png} \end{center}

\end{frame}

% slide 21
\begin{frame}[fragile]\frametitle{Conclusion}

\begin{center} Function theory is an important part of mathematics and of the world around us. It is used every day in applications like the stock market, to calculating monthly budgets, to rendering computer graphics. \end{center}
\begin{center}\includegraphics[scale=0.3]{Function_CompGraphics_slLast.jpg} \end{center}

\end{frame}

% slide 22
\begin{frame}[fragile]\frametitle{Works Cited}

\begin{itemize}
  \item http://mathworld.wolfram.com/Function.html
  \item http://functions.wolfram.com/ElementaryFunctions/
  \item http://oregonstate.edu/instruct/mth251/cq/FieldGuide/
  \item http://oregonstate.edu/instruct/mth251/cq/FieldGuide/linear/lesson.html
  \item http://math.stackexchange.com/questions/575198/what-purpose-does-the-use-of-functions-serve-in-mathematics
  \item http://math.stackexchange.com/questions/108133/what-are-functions-used-for
  \item http://math.feld.cvut.cz/mt/txtb/3/txeba3a.htm
\end{itemize}

\end{frame}

\end{document}