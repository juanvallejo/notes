\documentclass{beamer}
\usepackage{verbatim, graphicx}
\usetheme{Copenhagen}


\begin{document}
\begin{frame}
  \title{Math 301: Function Theory}
  \author{Juan Vallejo}
  \maketitle
\end{frame}

% slide 1
\begin{frame}[fragile]\frametitle{What Are Functions?}
\begin{center}\textbf{Definition} \end{center}

\begin{itemize}
  \item In mathematics, a function is a relation between a set of inputs and a set of permissible outputs with the property that each input is related to exactly one output. %No comments to display here
  
  \vfill
  \begin{center}\includegraphics[scale=0.5]{Function_IO_sl2.png} \end{center}
  
  \vfill
  \item General format of functions:
  
  \vfill
  \begin{equation*}
    f(x) = x^2
  \end{equation*}
  \vfill

\end{itemize}

\end{frame}

% slide 2
\begin{frame}[fragile]\frametitle{Why Functions?}
%\begin{center}\textbf{Definition} \end{center}

\begin{itemize}
  \item They are an integral part of mathematics and computational world
    \begin{itemize}
      \item Theory is foundation for functional programming paradigm seen in languages such as Haskell and Python
    \end{itemize}
  \item Allow for use of rules more than once in an expression
  \item Allow for different parameter values
  \item Useful for recognizing similarities in a set of observations

\end{itemize}

\end{frame}

% slide 3
\begin{frame}[fragile]\frametitle{Components $-$ Domain and Range}
%\begin{center}\textbf{Definition} \end{center}

Functions can be seen as relations that uniquely associate members of one set with members of another set. \textit{A function from $A \to B$} is an object $f$ such that every $\alpha \in A$ is uniquely associated with an object $f(\alpha) \in B$. A function is therefore a many-to-one relation.

\vfill
\begin{itemize}
  \item The set $A$ of values at which a function is defined is called its domain
  \item The set $f(A) \subset B$ of values that the function can produce is called the \textit{range}. It is denoted as follows:
  \begin{equation*}
    D(f) = \lbrace x \in \mathbb{R}; f(x) \rbrace
  \end{equation*}
  \item In other words, range of a function is the set of all values obtained by substituting arguments from its domain

\end{itemize}

\end{frame}

% slide 4
\begin{frame}[fragile]\frametitle{Components $-$ Real Functions}
%\begin{center}\textbf{Definition} \end{center}

A function is defined as ``Real" when $f: A \subset \mathbb{R} \in \mathbb{R}$
\vfill
\begin{itemize}
  \item Domain elements in real functions are mapped to unique range elements
  \begin{itemize}
    \item Can be expressed graphically by performing the \textit{vertical line test}
  \end{itemize}
  \item They are the most important type of \textit{mapping}.
  \begin{itemize}
    \item Can be thought of as any mapping from some subset of the set of real numbers to the set of real numbers
  \end{itemize}
  
\end{itemize}

\end{frame}

% slide 5
\begin{frame}[fragile]\frametitle{Components $-$ Real Functions}
\begin{center}\textbf{Basic Notions of Real Functions} \end{center}

\vfill
\begin{itemize}
  \item Can be thought of as a perscription which assigns values to arguments
  \begin{itemize}
    \item For example: $y = f(x)$ means that to the value $x$ of the argument, the function $f$ assigns the value $y$
    \item Second notation: $f:x \to y$ implies that the function $f$ sends the value $x$ to $y$
  \end{itemize}
  \item Usual way of specifying assignments is that the function value $y$ can be obtained by substituting $x$ to a specific formula:
  \begin{equation*}
    f(x) = 2x + 3
  \end{equation*}
  
\end{itemize}

\end{frame}

% slide 6
\begin{frame}[fragile]\frametitle{Components $-$ Real Functions}
\begin{center}\textbf{Substitution and Visualization} \end{center}

\vfill
\begin{center} $f(a) = f \vert_a  =  f \vert_{x=a}$ \end{center}

\vfill
\begin{itemize}
  \item Used when function needs adjustment, but not yet ready for substitution. Example:
  \begin{equation*}
    \left.\frac{x^2 - 1}{x-1}\right\rvert_{x=3} = \left.\frac{(x-1)(x+1)}{x-1}\right\rvert_{x=3} = 4.
  \end{equation*}
  \item Graphs are common way of visuaizing functions in mathematics.
  \item Used to mark in the two-dimensional plane $(x,y)$ all couples $x, f(x)$
  
\end{itemize}

\end{frame}

% slide 7
\begin{frame}[fragile]\frametitle{Types of Functions}

\vfill
\begin{center}\textbf{Functions are primary objects of study in calculus, consisting of various types:} \end{center}

\vfill
\begin{itemize}
\item linear
\item polynomial
\item trigonometric
\item logarithmic
\end{itemize}

\end{frame}

% slide 8
\begin{frame}[fragile]\frametitle{Linear Functions}
% \begin{center}\textbf{Substitution and Visualization} \end{center}

\vfill
Allow us to approximate more complicated functions in differential calculus. There are three standad forms of linear functions:

\vfill
\begin{align*}
  f(x) &= mx + b &\,\text{(Slope-intercept form)} \\
  y - y_o &= m(x - x_0) &\, \text{(The ``point-slope" form)} \\
  Ax + By &= C &\, \text{(The ``general" form)}
\end{align*}

\end{frame}

% slide 9
\begin{frame}[fragile]\frametitle{Linear Functions}
\begin{center}\textbf{Frequently used to solve for the following calculations:} \end{center}

\vfill
\begin{itemize}
  \item \textbf{Intercepts}
  \begin{itemize}
    \item Finding the slope, x-intercept, and y-intercept of the line given its equation
  \end{itemize}
  \item \textbf{Intersections}
  \begin{itemize}
    \item Finding the intersection of two lines from their equations
  \end{itemize}
  \item \textbf{Equations}
  \begin{itemize}
       \item Finding the equation of the line through a series of points given one of the following three pairs of data:
       \begin{itemize}
        \item the slope of the line and the y-intercept
        \item the slope of the line and a point $(x_0,y_0)$ on the line
        \item the coordinates of two points on the line
       \end{itemize}
  \end{itemize}

\end{itemize}

\end{frame}


% slide n
\begin{frame}[fragile]\frametitle{File Structure}

File is separated into frames: % AKA slides

\vfill

Inside the document:

\vfill
\begin{verbatim}

\begin{frame}\frametitle{Title}

Contents 

\end{frame} \end{verbatim}

\vfill

Compile to PDF! (DVI has \ldots issues)

\vfill


\end{frame}

\begin{frame}[fragile]\frametitle{Spacing}

\begin{itemize}
\item Text is automatically centered vertically.
\item Starts flush left.
\item No separation or indentation between paragraphs. %why do you have a paragraph?

\end{itemize}

He's a stiff! Bereft of life, he rests in peace! If you hadn't nailed him to the perch he'd be pushing up the daisies!
His metabolic processes are now history! He's off the twig!  He's kicked the bucket, he's shuffled off his mortal coil, run down the curtain and joined the bleeding choir invisible!

He's an ex-parrot!

%Quite crushed together


\end{frame}

\begin{frame}[fragile]\frametitle{Spacing}

\vfill

Vertical spacing: give blank lines before and after the command % not always necessary but at first, it's a good idea.
\begin{itemize}
  \item Manual vertical spacing: \verb#\vspace{amount in units}#  %many units, pts. inches, centimeters
  \item Automatic spacing: \verb#\vfill#
\end{itemize}

\vfill

Horizontal spacing: 
\begin{itemize}
  \item Manual vertical spacing: \verb#\hspace{amount in units}# 
  \item Automatic spacing: \verb#\hfill#
\end{itemize}

\vfill

\end{frame}

\begin{frame}\frametitle{Earlier example}

\begin{itemize}
\item Text is automatically centered vertically.
\item Starts flush left.
\item No separation or indentation between paragraphs. %why do you have a paragraph?
\end{itemize}



\end{frame}

\begin{frame}\frametitle{With vfill}

\vfill

\begin{itemize}
\vfill

\item Text is automatically centered vertically.

\vfill

\item Starts flush left.

\vfill

\item No separation or indentation between paragraphs. %why do you have a paragraph?


\end{itemize}
\vfill %Show them the code

\end{frame}

\begin{frame}\frametitle{vspace}
He's a stiff! Bereft of life, he rests in peace! If you hadn't nailed him to the perch he'd be pushing up the daisies!
His metabolic processes are now history! He's off the twig!  He's kicked the bucket, he's shuffled off his mortal coil, run down the curtain and joined the bleeding choir invisible!

He's an ex-parrot!
\end{frame}

\begin{frame}\frametitle{vspace}

He's a stiff! Bereft of life, he rests in peace! If you hadn't nailed him to the perch he'd be pushing up the daisies!
His metabolic processes are now history! He's off the twig!  He's kicked the bucket, he's shuffled off his mortal coil, run down the curtain and joined the bleeding choir invisible!

\vspace{0.1in}

He's an ex-parrot!
\end{frame}



\begin{frame}\frametitle{Overlays}

\vfill
Thinking of a frame as a slide in a presentation \ldots 

\vfill
\pause


Overlays are \pause slides \pause that \pause are added \pause as the presenter \pause speaks.  %Basically the file makes a series of pages in the PDF file, one page for each overlay.

\vfill

\pause Two general methods \ldots

\end{frame}

\begin{frame}[fragile]\frametitle{Pause: Quick and Dirty Method}

\vfill
\begin{itemize}
  \item Command: \verb#\pause# 
  \pause
  \vfill
  \item Insert where the overlay should occur.  \pause
  \vfill 
  \item Nothing complicated:\vfill  \pause
  \begin{itemize}
    \item No displayed math;\vfill  \pause
    \item Nothing to appear then disappear;\vfill  \pause
    \item Entire environments at a time.\vfill  
  \end{itemize}
\end{itemize} % Show them the code for this part

\end{frame}

\begin{frame}{onslide: far more robust}

Recall the Taylor series for $\sin$ and $\cos$:

\begin{align*}
  \cos(x) & = \sum_{n=0}^\infty \frac{(-1)^nx^{2n}}{2n!} \\
  \sin(x) & = \sum_{n=0}^\infty \frac{(-1)^nx^{2n+1}}{(2n+1)!} 
\end{align*}

\onslide<2->{If $x=i \theta $ where $i^2 = -1$, then:
\begin{align*}
\onslide<3->{  \cos(i\theta) & = \sum_{n=0}^\infty \frac{(-1)^n(i\theta)^{2n}}{2n!} 
\onslide<4->{ = \sum_{n=0}^\infty\frac{(-1)^n(-1)^n\theta^{2n}}{2n!} 
 = \sum_{n=0}^\infty\frac{\theta^{2n}}{2n!} \\
}
 % \sin(i\theta) & = \sum_{n=0}^\infty \frac{(-1)^n(i\theta)^{2n+1}}{(2n+1)!} 
 % \onslide<4->{ = \sum_{n=0}^\infty\frac{(-1)^n(-1)^ni\theta^{2n+1}}{(2n+1)!}
 %  = i\sum_{n=0}^\infty \frac{\theta^{2n+1}}{(2n+1)!} 
  }
\end{align*} }

\end{frame}

\begin{frame}[fragile]{onslide: syntax}

\vfill
\begin{itemize}
  \item Initial frame is considered slide 1.\vfill
  \item Content from slide $n$ onward: \verb#\onslide<n->{ Content }#\vfill
  \item Content on slides $m-n$: \verb#\onslide<m-n>{ Content }#\vfill
  \item Content on slides $i,j,k$: \verb#\onslide<i,j,k>{Content}#\vfill
  \item More options \ldots 
\end{itemize}


\end{frame}

\begin{frame}\frametitle{Graphics}

Here is a picture \ldots 

\begin{center}
\includegraphics[scale=0.5]{Trefoil.jpg}
\end{center}


\end{frame}

\begin{frame}[fragile]\frametitle{Here's how you get it \ldots}

\begin{itemize}
  \item In preamble: \verb#\usepackage{graphicx}# (accept no substitute) \vfill
  \item Generate graphic in appropriate format: pdf, jpg, gif, eps\vfill
  \item Command: \verb#\includegraphics[options]{filename}#\vfill
  \item Common options:\vfill
  \begin{itemize}
    \item scale=\textsl{scaling factor}\vfill
    \item rotate=\textsl{counterclockwise rotation in degrees}\vfill
    \item many others\vfill
  \end{itemize}
\end{itemize}
\end{frame}


\begin{frame}\frametitle{A Word about Compilation Errors\ldots}

\vfill

Get an error message while compiling, 
\pause

\vfill

it might as well say 

\pause

\vfill
\textsc{Bad Karma}.

\vfill

\end{frame}

\end{document}